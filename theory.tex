% $Id: introduction.tex 65669 2015-01-09 14:55:20Z tgershon $

\section{Theory}
\label{sec:Theory}

%Why loops no FCNC
%why suppressed
%Decay Rate formalism
%Make point of sum of diagrams

As mentioned previously \lhcb predominantly searches for new physics (NP) indirectly through measurements of observable quantities.  One of these quanitites is decay rates, which is the subject of the analysis presented in this report.  In general, the differential decay rate of a specific particle decay is given by,
\begin{equation}
  \label{eq:fgr}
  d\Gamma(P_A\to1 + ......n) = \frac{|\mathscr{M}|^2}{2M_A}dQ
\end{equation}

where $M_A$ is the mass of the decaying particle, $dQ$ is the lorentz invariant phase space (LIPS) and $|\mathscr{M}|^2$ is the matrix element for that decay \cite{halzen1984quarks}.  One can calculate the LIPS and insert it into equation \ref{eq:fgr}, giving a differential decay rate for a general $n$ body decay to be,
\begin{equation}
  \label{eq:diffdec}
  d\Gamma(P_A \to 1+ .....n) = \frac{1}{2M_A}\bigg(\prod_{f}^{N}\frac{d^3p_f}{(2\pi)^3(2E_f)}\bigg)|\mathscr{M}|^2(2\pi)^4\delta^4\bigg(\sum_{f}^np_f-p_A\bigg)
\end{equation}
where the subscript f represents a final state particle and A is the initial particle \cite{halzen1984quarks}.

The interesting part of equation {\ref{eq:diffdec}} is the matrix element, $|\mathscr{M}|$.  This contains all of the information about the particle physics that governs the probability of interactions taking place e.g couplings, propogators. For a decay of a particle $P_A$ to a given final state there is almost always more than one way in which the decay can proceed meaning more than one feynman diagram can be drawn for the given decay.  Each Feynman diagram has an associated decay amplitude that can be calculated using Feynman rules, which introduce a different factor for each propogator and vertex in the decay.  The total matrix element $\mathscr{M}$ is then given by,

\begin{equation}
  \label{eq:matrix}
  |\mathscr{M}|^2=|\sum_i A_i|^2
\end{equation}

where $A_i$ is the amplitude calculated from an individual feynman diagram.  The consequences of this are that every possible feynman diagram contributes to the observable decay rate, although some provide negligable contributions.

As flavour changing neutral currents are forbidden in the standard model, the only way a b \to s transistion can take place is via an electroweak loop.  An example of a minimal electroweak loop is shown in Figure \ref{fig:loop}.
\begin{figure}
  \centering
  \includegraphics[width=0.5\textwidth,angle=270]{Loop.pdf}
  \caption{A minimal electroweak loop showing a b \to s transistion}
  \label{fig:loop}
\end{figure}
Here, a virtual W- boson is created from the vacuum which changes the flavour of the b quark to any up type quarks which then recombines with the W- boson to create an s quark, giving an overall change from a b quark to an s quark.  Although theoretically it can be any of the up type quarks, the top quark provides the dominant contribution due to its large mass.

As the total decay rate is the sum of all amplitudes, as shown in equation \ref{eq:matrix}, the overall matrix element for an electroweak loop has to include all possible mediators of the loop.  If NP beyond the standard model exists, it could enter as a mediator of loop diagrams which would add another amplitude to the matrix element calculation.  Consequently, the total decay rate would be different and the standard model prediction would no longer be correct.  Therefore, by measuring decay rates (or in practice branching fractions) and comparing results to standard model predictions new physics can be indirectly searched for.

One of the main advantages of indirect searches is that they offer sensitivity to NP at energies higher than what is accessible through direct searches.  Also, they are sensitive to a wider range of NP effects rather than just the target of the search being performed.  However, loop decays are heavily suppressed in the standard model due to the extra vertex factors they introduce to the decay amplitude.  This makes loop decays rare, meaning large data samples and high detection efficiencies are needed to study them with any precision.








