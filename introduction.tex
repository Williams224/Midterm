% $Id: introduction.tex 65669 2015-01-09 14:55:20Z tgershon $

\section{Introduction}
\label{sec:Introduction}
The \lhcb experiment is one of 4 detectors making use of CERNs Large Hadron Collider (LHC). The LHC is a circular collider with a 27km circumference that accelerates protons and heavy ions to record breaking energies.  To date a center of mass energy of 13TeV has been achieved with protons.  The accelerated protons/ions collide at 4 interaction points around the circumference of the collider, each of which is surrounded by a detector.  There are two general purpose detectors at the LHC known as \atlas and \cms which have a wide range of physics aims.  The other two detectors, however, are specialised for more specific physics goals.  One of these detectors, ALICE, is specialised for studying heavy ion collisions.  \lhcb is specialised for studying particles containing beauty quarks in the forward region.

Also, out of the 3 experiments focused on studying proton-proton (pp) collisions, \lhcb is the only one not focused on direct searches for new physics.  Rather than searching for new particles directly created in the pp collisions, \lhcb searches for new physics by precisely measuring quantities that would be indirectly affected by the presence of physics beyond the standard model.  These measurements include (but are certainly not limited to): Rare decay rates, CKM angles, CKM matrix elements and CP asymmetries.

The analysis section of this report studies the decay \Bd \to \Kstar \etaz, which is required for the to search for the rare decay \Lb \to p \Km \etaz.  In general, decays that are forbidden at tree level in the standard model can only proceed via loop diagrams.  As theoretically a large range of particles can contribute to these loop diagrams, any new physics could contribute and change the rate at which this process happens.  By measuring branching fractions and comparing these to standard model predicitons, new physics can be searched for.

This report also details work that has been carried out to validate the simulation software used at \lhcb. As described in section \ref{sec:Simulation} it is impossible to perform a modern particle physics analysis without simulating the production and detection of relevant particles and processes.  Extensive work is carried out to tune the element of the software that generates particles with correct distributions, but this report presents validation tests of the way interations with the detector materials are simulated.

%LHC- very basic

%LHCb- Goals
%CKM angle measurements
%rare decays
%CP violation
%lepton universality

%LHCb-Achievements

%LHCb-
