% $Id: introduction.tex 65669 2015-01-09 14:55:20Z tgershon $

\section{Simulation}
\label{sec:Simulation}
It is crucial to any particle physics experiment to understand and be able to calculate the behavoiur of the detector and apparatus being used.  No detector is perfect, therefore only a fraction of the particles created in the process being studied actually get fully measured and reconstructued.  From this subset of detected particles one has to infer the total number of particles created and any relevant distributions related to the measurement in question, the only practical way to calculate these detector efficiencies is with a full detector simulation.  A full detector simulation is also crucial for detector performance studies, understanding backgrounds and comparisons between theory and experiment.

The complexity of particle physics detectors and the many degrees of freedom in the underlying problem behind a detector simualtion make it unfathomably complex and highly multi-dimensional.  Consequntly, the ``monte carlo''(MC) or statistical sampling method is used to perform these simulations due to the fact it deals with multi dimensinal problems very well and it makes the problem conceptually far easier to understand.  More specifically direct monte carlo is used.

The first step in the monte carlo simulation is the generation of a pp interaction.  This is the responsibility of the event generator, which generates particles with randomly generated properties sampled from the correct distributions.  In \lhcb \pythia is the defualt generator used for general purpose production, but \herwig++ is sometimes used as a cross check\cite{Sjostrand:2007gs,Herwig}.  This is because \herwig++ implements a different hadronisaton mechanism to \pythia.  There are also some specialised generators used which include: \bcvegpy for the production of \Bc mesons, GenXicc for \Xic production and Hiijing for heavy ion collisons \cite{bcvegpy,Gyulassy:1994ew,Chang:2007pp}.