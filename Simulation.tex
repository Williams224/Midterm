% $Id: introduction.tex 65669 2015-01-09 14:55:20Z tgershon $

\section{Simulation}
\label{sec:Simulation}
It is crucial to any particle physics experiment to understand and be able to calculate the behavoiur of the detector and apparatus being used.  No detector is perfect, therefore only a fraction of the particles created in the process being studied actually get fully measured and reconstructued.  From this subset of detected particles one has to infer the total number of particles created and any relevant distributions related to the measurement in question, the only practical way to calculate these detector efficiencies is with a full detector simulation.  One area of the \lhcb physics programme where detector simulation is of particularly great importance is semi leptonic decays involving neutrinos.  Due to the \lhcb detector not being hermetic neutrinos escape the detector, consequently the only way to perform analyses involving neutrinos is with template fits based on monte carlo predictions.  An example of an analysis exploiting this strategy is the measurement of \Vub using \Lb $\rightarrow$ \proton \muon $\nu$ decays, which is a result considered to be of high importance \cite{LHCb-PAPER-2015-013}.  A full detector simulation is also crucial for detector performance studies, understanding backgrounds and comparisons between theory and experiment.

The complexity of particle physics detectors and the many degrees of freedom in the underlying problem behind a detector simualtion make it unfathomably complex and highly multi-dimensional.  Consequntly, the ``monte carlo''(MC) or statistical sampling method is used to perform these simulations due to the fact it deals with multi dimensinal problems very well and it makes the problem conceptually far easier to understand.  More specifically, direct monte carlo is used.

The first step in the monte carlo simulation is the generation of a pp interaction.  This is the responsibility of the event generator, which generates particles with randomly generated properties sampled from the correct distributions.  In \lhcb \pythia is the defualt generator used for general purpose production, but \herwig++ is sometimes used as a cross check\cite{Sjostrand:2007gs,Herwig}.  Both of these generators also simulate the hadronisation of the particles created in the hard pp interaction but \herwig uses a cluster model and \pythia uses a string model.  There are also some specialised generators used which include: \bcvegpy for the production of \Bc mesons, GenXicc for \Xic production and Hiijing for heavy ion collisons \cite{bcvegpy,Gyulassy:1994ew,Chang:2007pp}.
%% TODO ADD EVTGEN

The next stage of the simulation process is the transportation of the generated particles through the detector and all the interactions with the detector material that occur.  In \lhcb this is performed by the \geant toolkit which is a well established software framework solely created to simulate the interatctions of particles with matter \cite{Agostinelli:2002hh}.  It is this stage of the simulation process, as performed by \geant, that is the subject of study in this report.  \geant has the capability to simulate all relevant hadronic, electromagnetic and optical interactions in a wide energy range for many long lived particles. Futhermore, \geant manages the geometry of the apparatus being simulated and also has the capbaility to simulate magnetic fields that are present at all collider based particle physics experiments.

\geant was first commisioned in 1993 as a result of studies that looked at how modern computing techniques could impove the existing fortran based GEANT frameworks that date back to 1974\cite{Agostinelli:2002hh,Brun:118715}.  These studies lead to \geant being implemented as an object orientated framework in C++.  One of the many advantages of object orientation is the ease by which improvements and refinements to the existing code can be made.  Consequently, there are regular new releases of \geant, the latest version being 10.1\cite{g410.1rn}. However, this is yet to be implemented in \lhcb; until recently \geant v9.5.p02 was implemented in the \lhcb simulation package but this has recently been updated to v9.6.p04.  When a new release of \geant is adopted into the \lhcb simulation package it is important to validate the reuslts of the new package and understand any discrepancies compared to the previous version.  If changes to simulation results are not understood this could have series consequences for physics analyses due to the unavoidable reliance on simulation for detector efficiencies etc.  Part of the study presented in this report is a comparison of electromagnetic physics results between \geant version 9.5.p02 and version 9.6.p04 for two seperate scenarios, both of which are important to \lhcb.

The object orientated design of \geant means that it is also very easy to have a variety of models avaiable for each interaction process; it is well known that the model giving the optimum agreement with data varies for different energy ranges and different particle types.  Consequently, it is important to be able to tailor the simulation to the specific scenario under investigation.  However, with the large number of interaction processes simulated and the large array of long lived particles to consider even a small selection of models for each process and particle would quickly lead to a very extensive range of possible configurations for the same scenario.  Therefore, \geant provides a set of physics lists (PL) each of which specifies a fixed set of models designed to be optimal for a smaller range of scenarios.  For example, there are PLs available for low energy scenarios which would specify models that show the best agreement with data at lower energy.  The effect of PL choice for scenarios relevant to \lhcb also forms part of the study detailed in this report.

\paragraph{Production Cuts} When a detector simulation takes place many secondary particle are created through processes such as bremstrahlung and pair production.  Each of these particles then has to be tracked and simulated by \geant which takes a non negliagable amount of CPU time. However, a lot of these particles are low energy particles that contribute very little to the overall result of the simulation.  Therefore, a minimum range cut is applied in \geant in order to minimise the amount of CPU time that is wasted tracking and simlutaing particles that have a negligible effect on the end result.  This requires any secondary particles created to travel a minimum distance for them to be simulated by \geant.  In \lhcb this cut has been chosen as 5mm, a distance that has been found to reduce CPU time sufficiently but have a small enough impact on the simulation result.  There are however studies on going to investigate whether changing range cuts in certain sub detectors can improve agreement between data and MC without an unfeasible increase in CPU time %Cite jimmys thesis.

\subsection{Simple Calorimeter Test}
\label{sec:Simple Calorimeter Test}
The first test that has been setup is an extended electromagnetic \geant example designed to model an electromagnetic calorimeter.  This has been setup to model the \lhcb electromagnetic calorimeter (ECAL) as closely as possible. As described in section \ref{sec:Detector}, the \lhcb ECAL is a sampling calorimeter consisiting of 66 alternating layers of lead and plastic scintillator \cite{LHCb-TDR-002}.  This example has the ability to change: layer thicknesses, number of layers, lateral size and crucially the physics list that is used. All of these parameters can be changed with the use of a \geant macro file, meaning they can be changed easily without recompilation. 

%\subsubsection{Calorimetry Background}
%\label{sec:Calorimetry Background}
\subsubsection{Comparison of \geant version 9.5.p02 and 9.6.p04}
The basic principle behind this comparison was to run this example within both \geant versions and compare selected results that could have implications for \lhcb.  Firstly, checks were made to ensure that the implementation of the example was unchanged between versions to ensure any discrepencies could be attributed to changes global to the \geant framework and not just a change in this specific example.  In order to make this a direct comparison between what is currently implemented and what will be implemented the example was customised to run with the \lhcb private PL.  Therefore, the same physics models are being used in both versions, but the models themselves could have changed between versions.

\paragraph{Fractional Resolution Investigation}
\label{sec:Fractional Resolution Investigation}

The first result used as a comparison tool between \geant versions was the fractional resolution of the electromagnetic calorimter.  No detector is perfect, therefore the energy of an incident particle is measured with a resolution defined by a gaussian distirbution.  However, the width of this gaussian distribution varies with the energy of the incident particle.  The fractional resolution is defined as $\frac{\sigma(E)}{E}$ where $\sigma(E)$ is the standard deviation of a particles measured energy at a given incident energy E.  The variation of the fractional resolution as a function of energy is given by, 
\begin{equation}\label{eq:fr}
\frac{\sigma}{E}=\frac{a}{\sqrt{E}} \bigoplus \frac{b}{E} \bigoplus c
\end{equation}
where a, b and c are constants to be defined\cite{wigmans2000calorimetry}.  

The 'a' term in equation \ref{eq:fr} arises from statistical fluctuations in the electromagnetic shower induced by the sampling calorimeter.  In an electromagnetic shower many particles are produced and the energy measured by the ECAL is the sum of the energies deposited by each particle.  However in a sampling calorimeter only a fraction of the shower takes place in active regions, therefore only a fraction of the shower particles are actually measured.  Consequently the number of shower particles that are measured is subject to poisson fluctutations with a standard deviation of $\sqrt{N}$, where N is the number of shower particles in the active region.  Assuming the calorimeter is linear (which it should be) the number of shower particles produced is proportional to energy, therefore $\frac{\sigma}{E} \propto \frac{\sqrt{N}}{N} \propto \frac{1}{\sqrt{E}}$. In a sampling calorimeter it is sampling fluctiations that dominate the resolution, thus the 'a' term is usually by far the largest term in equation \ref{eq:fr}.  

The 'b' term in equation \ref{eq:fr} arises due to electronic noise which causes energy independent resolution effects.  However, this is not simulated in this study and will therefore not be considered.  The 'c' term in equation \ref{eq:fr} arises due to shower leakage from the ECAL.  At a given energy the amount of energy lost from the ECAL is subject to event by event fluctuations, which leads to broadening of the resolution.  However, the total amount of energy lost from the ECAL is proportional to the energy of the incident particle and this dominates over the event by event effects.  This leads to the standard deviation of measured energies due to shower leakage being proportinal to incident energy, and consequently energy indpendent for the fractional resolution.

The first step towards obtaining the fractional resolution of the model ECAL was to fire electrons into the ECAL at several fixed energies.  At each energy a histogram was produced showing the distribution of energy deposited in active regions of the ECAL per event.  This distribution is predicted to be gaussian, therefore a minimum chi squared fit of a gaussian function was performed at each energy to obtain the mean and standard deviation of the distribution. An example of one of these fits is shown in figure \ref{fig:Gauss}.  This was performed at 13 different energies between 1.78GeV and 44.44GeV, this range was chosen because it includes the majority of electron energies expected at \lhcb.

\begin{figure}[h!]
  \centering
  \includegraphics[width=\textwidth]{GaussExample.pdf}
  \caption{The distribution of energy deposited in scintillator(active) layers at 25 GeV}
  \label{fig:Gauss}
\end{figure}

From these fits the standard deviation divided by the mean was plotted against the incident particle energy, as defined in the configuration of \geant.  $\frac{\sigma}{E}$ is expected to follow equation \ref{eq:fr} but without the b term. Therefore a mimimum chi squared fit to,
\begin{equation}
  \label{eq:fitfr}
  \frac{\sigma}{E}=\frac{a}{\sqrt{E}}\bigoplus c
\end{equation}
was performed with the parameters 'a' and 'c' left floating as these are to be determined by the fit.  This process was applied to both \geant version 9.5.p02 and 9.6.p04 with the same energies and 10000 events at each energy.  The results of these fits for both \geant versions are shown in figures \ref{fig:straightres}, \ref{fig:coolres} and table \ref{tab:results}.
\begin{figure}[h]
  \centering
  \includegraphics[width=\textwidth]{StraightCompareUse.pdf}
  \caption{Comparison of observed fractional resolution in \geant v9.5 and v9.6}
  \label{fig:straightres}
\end{figure}
\begin{figure}[h]
  \centering
  \includegraphics[width=\textwidth]{CurvedCompare.pdf}
  \caption{Comparison of observed fractional resolution in \geant v9.5 and v9.6}
  \label{fig:coolres}
\end{figure}

\begin{table}[h]
  \centering
  \begin{tabular}{|c|c|c|c|}
      \hline
      Version: & v9.5 & v9.6 & Difference  \\ \hline
      A term    & 0.0952$\pm$0.0003 & 0.0922$\pm$0.0003  & 8.43$\sigma$, 3.3\% \\ \hline
      C term    & 0.0051$\pm$0.0003 & 0.0049$\pm$0.0003 & consistent \\ \hline
      $\frac{\chi^2}{ndf}$   &1.798  & 1.048 &  \\ \hline
  \end{tabular}
  \caption{Fractional resolution results for comparison of \geant versions}
  \label{tab:results}
\end{table}

It is clear from these fits that there is a significant discrepancy in the statistical term of the fractional resolution parameterisation between \geant versions, despite the same PL being used in both versions.  The simulation of electromagnetic showers is a well established process that, in general, shows good agreement with data.  Therefore, progression of results this significant is not expected.  As many different processes contribute to electromagnetic showering more investigations were needed to understand exactly which part of the simulation is giving rise to this discrepancy.

\paragraph{Sampling Fraction and Shower Profile Investigations}
\label{sec:Sampling Fraction and Shower Profile Investigations}
In order to understand the unexpected results described in section \ref{sec:Fractional Resolution Investigation} it was decided that a study of the shower profiles in both active and passive layers of the calorimeter would be performed.  A shower profile is a plot of dE/dX against X, where X is the longitudinal depth from the front face of the calorimeter and E is the energy deposited.  Although X is often expressed in terms of radiation lengths, in this case the layer number was used because the shower profiles were in bins corresponding to the thickness of one layer of the ECAL.

Again, electrons at 13 different energies between 1.56GeV and 44.44GeV were fired into the ECAL using both \geant versions.  At each energy, shower profiles for active and passive layers were considered seperately and the profiles resulting from each version of \geant were plotted on top of each other for comparison.  The results at 1.56GeV and 44.44 GeV are shown in figures \ref{fig:1.56sp} and \ref{fig:44.44sp} respectively, all other energies are shown in \ref{ap:sp}.
\begin{figure}[h]
  \begin{subfigure}{.5\textwidth}
    \centering
    \includegraphics[width=\linewidth]{ShowerProfileLead44.pdf}
    \caption{Normalised shower profile for electrons in Lead at 44.44GeV}
    \label{fig:44.44splead}
  \end{subfigure}
  \begin{subfigure}{.5\textwidth}
    \centering
    \includegraphics[width=\linewidth]{ShowerProfileScintillator44.pdf}
    \caption{Normalised shower profile for electrons in Scintillator at 44.44GeV}
    \label{fig:44.44spscint}
  \end{subfigure}
  \caption{Normalised shower profiles}
  \label{fig:44.44sp}
\end{figure}
\begin{figure}[h]
  \begin{subfigure}{.5\textwidth}
    \centering
    \includegraphics[width=\linewidth]{ShowerProfileLead178.pdf}
    \caption{Normalised shower profile for electrons in Lead at 1.78GeV}
    \label{fig:44.44splead}
  \end{subfigure}
  \begin{subfigure}{.5\textwidth}
    \centering
    \includegraphics[width=\linewidth]{ShowerProfileScintillator178.pdf}
    \caption{Normalised shower profile for electrons in Scintillator at 1.78GeV}
    \label{fig:44.44spscint}
  \end{subfigure}
  \caption{Normalised shower profiles}
  \label{fig:44.44sp}
\end{figure}
%PUT SHOWER PROFILES HERE
It is clear from these shower profiles that \geant version 9.6.p04 is depositing more energy in active layers and less energy in passive layers compared to \geant 9.5.p02.  This pattern is observed at all energies simulated.

To further investigate this behaviour the shower profiles were split by particle type; seperate shower profiles were plotted for electrons, positrons and photons.  These results can be seen in figures \ref{fig:1.56spseperate}  and \ref{fig:44.44spseperate}.
%PUT SEPERATE SHOWER PROFILES HERE
These reuslts show that the observed discrepancy can not be attributed to the way in which photons are modelled because the total energy deposited by photons is negligable and would therefore not be able to influence the overall shower profiles.

In order to quantify further this observed change in behaviour, the ECAL sampling fraction as a function of energy was studied for both \geant versions.  The sampling fraction, in this context, is defined as the energy deposited in active layers (visible energy) divided by the total energy deposited in the ECAL.  The total energy deposited in each type of material was obtained by integrating the shower profiles over all layers.  The resulting sampling fractions as a function of energy are shown in \ref{fig:sfcomp}. To further this quantification the ratio of energy deposited in active layers to passive layers was also plotted as a function of energy for both \geant versions and the results are shown in figure \ref{fig:ratiocomp}.  The results from both of these plots are summarised in table \ref{tab:sfcomp}.

\paragraph{Discussion of results for \geant version comparison}
\label{sec:Discussionofresultsone}
The results of these sampling fraction studies are significant, both statistically and in terms of the implications for the next simulation package used by \lhcb.  It is very clear that the sampling fractions have changed between \geant versions and this change would explain the observed discrepancy in fractional resolution seen in section \ref{sec:Fractional Resolution Investigation}.  Sampling more energy would always improve the fractional resolution of an ECAL because it means any statistical fluctutations are relatively less significant. The ECAL has to be calibrated in monte carlo such that an observed value of visible energy corresponds to the correct incident energy.  Since these results show that the visible energy of the ECAL has changed in this new version of \geant this calibration will have to be performed again when the new simulation package is released.

The precise cause of these discrepancies needs to be understood because there could be implications for the simulation of other detector components.  Therefore, the models used when the \lhcb PL is specified were compared between versions and the following changes were consider possible causes of the discrepancy:
\begin{itemize}
\item \textit{New implementation of UrbanMsc95 multiple scattering model.}  It is detailed in the release notes that sampling of multiple scattering in version 9.6 is now performed before the energy loss of each particle.  However, in version 9.5 and before energy loss was sampled before multiple scattering.  If this changed the scattering andlge distribution of electrons and positrons in the calorimeter, it could change the number of particles that scatter over the boundary between active and passive layers.
\item \textit{Change in angular distribution model for electron bremsstrahlung.} In version 9.6 a boosted dipole approximation is used for the angular distribution of photons emitted in electron bremmsstrahlung, as described in \cite{:/content/aip/journal/apl/80/17/10.1063/1.1473684}.  A change in the angular distribution of bremsstrahlung could also result in particles previously depositing energy in passive layers traversing the boundary between layers and depositing energy in active layers instead.
\end{itemize}

After contacting the \geant collaboration, the cause of the observed discrepancy is believed to lie with the changes to the multiple scattering model.  In fact, the \geant collaboration recommended that the multiple scattering models used in the \lhcb private PL are changed from \textit{UrbanMsc95} to \textit{UrbanMsc93} below 100MeV and \textit{WentzelVI} above 100MeV.  This would bring the simulation of multiple scattering within the LHCb private PL inline with the \geant \textit{emstandard opt1} physics list.

\subsubsection{Comparison of Physics Lists}
As the \geant collaboration had recommended a change to the LHCb private PL it was decided a thorough investigation of other PL options would be carried out using the simple calorimeter test.  This would be done using \geant version 9.6.p04, so that investigations directly related to the next version of the \lhcb simulation package could be carried out.  The results of these investigations would help to inform the decision of which PL to use for the next release of \gauss.  


The following physics lists supplied by \geant were investigated:
\begin{itemize}
  \item \textit{emstandard option1}
    This physics list is designed for HEP experiments but is described as \cms focused.
\end{itemize}




%Simple Calorimeter Test
%----Calorimetry Background
%----Comparison of Geant4 versions
%---------Resolution investigations
%---------Sampling fraction investigations
%---------Geant4 version comparison results discussion
%----Physics List investigations
%---------Fractional Resolution
%---------Sampling Fraction Results
%---------Results Discussion

%MSC
%--results


