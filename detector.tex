% $Id: introduction.tex 65669 2015-01-09 14:55:20Z tgershon $

\section{Detector}
\label{sec:Detector}

The \lhcb detector is a unique and specialised apparatus that has been optimised to meet the specific physics goals discussed in section \ref{sec:Introduction}, which largely involve the study of mesons and baryons containing beauty and charm quarks.  The resulting design is a single arm forward spectrometer which covers the pseudorapidity range $2 < \eta < 5$.  This limited angular coverage is motivated by the polar angle distribution of  $b$ and $\bar{b}$ quark production peaking at small angles to the beam pipe in both the forward and backward direction, as shown in figure \ref{}.  The consequence of this angular distribution is that at 8 TeV approximately $25\%$ of $b$ or $\bar{b}$ quarks are produced within the angular acceptance of the \lhcb detector, despite the geometrical acceptance of the detector being only $~0.09$ steradians ($0.7\%$ of a full sphere).

%Luminosity
Another unique characteristic of the \lhcb detector is the lower luminosity that it operates at.  During run 1 the ATLAS and CMS detectors operated at a luminosity

%General needs of detector 

%Tracking

%Calorimetery

%Particle identification

%Muons