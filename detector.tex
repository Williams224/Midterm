% $Id: introduction.tex 65669 2015-01-09 14:55:20Z tgershon $

\section{Detector}
\label{sec:Detector}

The \lhcb detector is a unique and specialised apparatus that has been optimised to meet the specific physics goals discussed in section \ref{sec:Introduction}, which largely involve the study of mesons and baryons containing beauty and charm quarks.  The resulting design is a single arm forward spectrometer which covers the pseudorapidity range $2 < \eta < 5$.  This limited angular coverage is motivated by the polar angle distribution of  $b$ and $\bar{b}$ quark production peaking at small angles to the beam pipe in both the forward and backward direction, as shown in figure \ref{}.  The consequence of this angular distribution is that at 8 TeV approximately $25\%$ of $b$ or $\bar{b}$ quarks are produced within the angular acceptance of the \lhcb detector, despite the geometrical acceptance of the detector being only $~0.09$ steradians ($0.7\%$ of a full sphere).

%Luminosity
Another unique characteristic of the \lhcb detector is the lower luminosity that it operates at.  During run 1 the \atlas and \cms detectors made use of a peak luminosity of $7\times 10^{33}cm^{-2}s^{-1}$ at the start of a fill which gradually decreased over the 6-8 hour lifetime of the fill as collisions took place.  However, \lhcb ran at a lower, levelled, luminosity of $4\times 10^{32}cm^{-2}s^{-1}$.  The process of luminosity levelling (at \lhcb) involves offestting the LHC proton beams in the transverse direction to reduce the probability of interaction \cite{Follin:2014nva}. As the luminosity of the \lhc decreases over the lifetime of the fill, the beam offset at \lhcb is decreased to keep the instantaneous luminosity delivered at \lhcb constant.  The result of this lower luminosity is only a mean number of proton-proton interactions per bunch crossing of 2.5 compared to 20 as experienced by the general purpose LHC detectors.  Consequently, the radiation hardness requirements of the \lhcb detector are reduced and backgrounds from other interactions in the same bunch crossing are also significantly lower.  This lower level of background is very advantageous for rare decay searches such as \Bz \to \muon \muon and various other precision physics goals of \lhcb.

\subsection{Tracking}
\label{sec:Tracking}


%Calorimetery

%Particle identification

%Muons