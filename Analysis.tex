% $Id: introduction.tex 65669 2015-01-09 14:55:20Z tgershon $

\section{Investigation of $\Bd \rightarrow \Kstar(892) \etaz$}
\label{sec:Analysis}
%motivation and stuff here TODO


The intention of this analysis is to use the full $3fb^{-1}$ dataset collected by \lhcb in run I of the LHC.  However, the $1fb^{-1}$ of data collected at a 7TeV centre of mass energy (CoM) and the $2fb^{-1}$ of data collected at an 8TeV centre of mass energy will be treated seperatley at least for the selection stage of this analysis.  This is because the differences in kinematic variables between these two CoM energies would make a common selection for both sub-optimal. So far only the $2fb^{-1}$ of 8TeV data has been considered, therefore all results in this report are based on that dataset.
\subsection{Decay Reconstruction}
\label{sec:Decay Reconstruction}
The \Kstar will be reconstructed in the channel $\Kstar \to \Kp\pim$ which has a branching fraction of $99.901\pm0.009\%$ \cite{PDG2014}.  The \etaz will be reconstructed in the channel $\etaz \to \pip\pim\piz$ where the \piz decays to 2 photons, which has a combined branching fraction of $21.83\pm0.28\%$\cite{PDG2014}.  This is not the dominant decay mode of the \etaz particle but the 2 decay modes with higher branching fractions contain only neutral particles in the final state.  This means there is no tracking information available and the \lhcb calorimeter (unlike the \atlas calorimeter) does not provide any directional information for the photons it detects.  Consequently, reconstruction efficiencies for all neutral final states are significantly lower and backgrounds are higher, which makes the \pip\pim\piz channel the optimal choice.  


Each of the final state particles in the decay chain are assigned a mass hypothesis based on information from all of the \lhcb sub detectors before being combined to form their respective mother particles.  The properties of the mother particle (e.g Energy, momentum, flight distance) are then determined with a fit to the daughter particles;   the $\chi^2$ of this fit can then aid the choice of candidate decays.

The combination of particles to form a \Bd candidate described is all done by the \lhcb reconstruction packages and the initial, loose, selection of \Bd candidates is based on this reconstruction. However, a better mass resolution can be achieved by refitting the entire decay tree of \Bd candidates whilst applying extra constraints.  This technique was first used by the \babar and is practically implemented with a Kalman fitter\cite{Hulsbergen:2005pu}.  Most commonly, any intermediate daughter particles are constrained to have their known masses and all tracks are constrained to come from the position where the hard pp interaction took place (primary vertex).  In this analysis all daughter particles are constrained to come from the primary vertex and the \etaz mass is constrained to the current world average ($547.826\pm0.018MeV$)\cite{PDG2014}.  The \Kstar mass is not constrained because it is a wide resonance; the natrual width of the \Kstar is around 50MeV.  It is impossible to reconstruct a particle with a better resolution than what is dictated by it's natural width.  Consequently,, one can expect correctly reconstructed \Kstar particles to still have a mass a considerable distance from the PDG value.  Therefore, constraining the \Kstar candidate to its PDG value can bias the refitted \Bd mass.

\subsection{Event Selection}
\label{sec:Selection}


The first stage of any analysis is selecting a subset of the \lhcb dataset that contains the 


%Decay Reconstruction
%Selection
%---Stripping
%---Trigger Selection
%---BDT
%---

